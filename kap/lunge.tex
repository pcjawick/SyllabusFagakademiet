\chapter{Lungesykdommer}
		%\chapterprecishere{Vuuuu, vuu vuu\par\raggedleft--- \textup{Molly}, En husky hund fra Drammen}
			Her presenteres lungesykdommene vi snakker om i kurset. Målet er å lære om KOLS. Forskjellen på KOLS og astma. Lungebetennelse, blodpropp og andre tilstander i lungene beskrives også, men kort.
		\section{Anatomi}
			\paragraph{Stor overflate\\}
				\begin{figure}[ht]
                      \centering
                      	\frame{\includegraphics[width=5in]{./kap/bilder/hjerte}}%!!! må byttes ut, copyright Greys anatomy
                      \caption{Et lungebilde}
                      {Her ser vi et bilde som illusterer lungene og den antomiske oppbygningen}%\textit{tjenestetilbudene}.]
                    \end{figure}
            \paragraph{300 millioner\\}
            	En enorm overflate er nødvendig for effektiv utveksling av oksygen og carbondioksid. Lungene er delt i lapper. Gasutvekslingen finner sted i alveolene(som det finnes 300 millioner av). Lungene har samme overflate som en tennisbane.
		\section{Fysiologi}
				\paragraph{Syre- base\\}
					CO\textsubscript{2} påvirker sammen med bicarbonat syre- base i lungene. Det er en av grunnene til at det tas blodgass av KOLS pasienter. Det kan også bidra til at pasienter med lungesykdommer kan være ustabile.
				\paragraph{Litt om O\textsubscript{2}\\}
					Oksygen kommer inn i kroppen via lungene. Det er ingen andre veier inn. 
		\section{Patologi}
			\paragraph{Arrvev\\}
				Det oppstår arrvev og skader som ødelegger alveolene (emfysem) og obstruksjon av luftveiene hinder luftpassasje i de større luftveiene. Sammen gjør dette at overflaten blir mindre.
			\paragraph{Emfysem\\}
				Konsekvensen av arrvev i lungene kan være store deler av lungen som forsvinner og store hulrom med luft er alt som er igjen. 
					\begin{figure}[ht]
                      \centering
                      	\frame{\includegraphics[width=5in]{./kap/bilder/hjerte}}%!!! må byttes ut, copyright Greys anatomy
                      \caption{Et CT-bilde av en emfysemlunge}
                      {Her ser vi et CT-bilde av lunger med emfysem. Dette er ikke alltid mulig å se på vanlig røntgen. Det samme gjelder med ultralyd.}%\textit{tjenestetilbudene}.]
                    \end{figure}
            \paragraph{Andre problemer\\}
            	Mange andre tilstander påvirker lungefunksjonen. Dette er ikke en fullstendig liste men: pneumothoraks, pleuravæske, lungebetennelser, lungeødem, lungearterieemboli og mange flere. Ofte er sykdomstilstanden kjent, men det er viktig å huske på at for eksempel KOLS pasienter har hyppigere pneumothoraks enn den vanlige befolkningen. Lærepoenget er at vi skal huske på at vi må gjøre diagnostikk, selv på velkjente KOLS pasienter med forverringer. 
		\section{Klinikk}
			\subsection{KOLS}
			\subsection{Pneumoni}
			\subsection{Lungeemboli}
		\section{Pasienteksempler}
			\subsection{Pasient 5}
			\subsection{Pasient 6}