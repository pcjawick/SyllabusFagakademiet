	\chapter{Diabetes i alderdommen}%diabetes og hypoglykemi
		\section{Punkter å ha med seg:}
			\begin{itemize}
				\item Diabetes hos eldre er resultat av for mye sukker i mange år.\\
				\item Hypoglykemi er den eneste metaboliske sykdommen som kan nevrologiske symptomer på bare en side.\\
				\item Proteiner ødelegger alt!\\
				\item Ikke glem at alle skal følges opp med øynene hvert år.\\
				\item HbA\textsubscript{1C} forteller noe om blodsukker belastningen: over 8 betyr diabetes. Jo høyere den er jo mer risiko for skade. Legg på 33\% for å finne gjennomsittsverdien.\\
				\item Hvis pasienten har 5 mg/dl i blodsukker om kvelden og skal ha 15 IE Lantus\textregistered. så virker ikke Lantus før om morgenen etter.\\
			\end{itemize}
		\section{Anatomi og fysiologi}
			Sukker tas opp i tarmen, og ender i blodbanen. Insulin lages i bukspyttkjertelen(=pankreas), og styrer cellenes evne til å ta opp sukker ved å koble seg til cellene(viktig begrep: insulinreseptor) som en liten minnepinne med mye informasjon. Når cellene påvirkes av insulin skjer det masse, men sukkeret i blodbanen reduseres.
		\section{Patologi}
			\paragraph{Insulinresistent\\}
				Til slutt blir cellene lei av alt insulinet og reagerer med å ikke være mottakelige lenger. De er utslitte av all lagringen av sukker. Insulinreseptorene blir mye ferre. På den måten forsvinner ikke sukker fra blodbanen lenger men svirrer rundt i hele kroppen i lengre tid.
			\paragraph{Tjuvkobling\\}
				De kobler seg på alle steder de kan og ødelegger kroppen proteiner ved å glykolysere dem. Det betyr at en eller flere sukker kobler seg til allerede eksiterende proteiner. Dette ødelegger proteinene i det organet de befinner seg. I leveren får vi fettlever. I blodårene ødelegges den fleksible åreveggen, og i øynene blir netthinnen slørete.
		\section{Klinikk}
			\subsubsection{En tøff medisin}
				Det er bare de mest virksomme medisiner som er helt pyton å ta. Mosjon, og endring av kosthold er de viktigste tiltakene for en pasient med diabetes type II, og er veldig virksomme ettersom mange sliter med å følge rådene. Dignosen stilles ved måling av blodsukker (enkeltverdi over 10, verdi over 8 mer enn to timer etter siste måltid. Alle skal ta glukosebelastningstest)
			\subsubsection{Mange sykdommer følger i kjølvannet}
		\subsection{Medikamenter\cite{legevakthandboka}}
				\subsubsection{Settes i huden(s.c.):}
					\paragraph{Hurtig-/korttidsvirkende insulinpreparater (Insulin Actrapid\textregistered, Insuman Rapid\textregistered):\\}Virker etter omlag 30 minutter, maksimal effekt etter 1–3 timer. Virkningen er over etter 8 timer. Brukes før maten.

					\paragraph{Ekstra hurtigvirkende insulinanalog (Humalog\textregistered, NovoRapid\textregistered og Apidra\textregistered):}Virker  umiddelbart, kan brukes i det man begynner å spise. Virkningen er over etter 3–5 timer.

					\paragraph{Middels langtidsvirkende insulin (Humulin NPH\textregistered, Insulin Insulatard\textregistered og Insuman Basal\textregistered):\\}Begynner å virke etter 1,5 timer. Maksimal effekt etter 4–8 timer. Tar 20–24 timer før effekten er over.

					\paragraph{Langtidsvirkende insulin (Lantus\textregistered, Levimir\textregistered):\\}Maksimal effekt etter 6–8 timer. Etter 24 timer er virkninger over.

					\paragraph{Kombinasjon av hurtig- og middels langtidsvirkende insulin (Insulin Mixtard\textregistered, Insuman Comb\textregistered, Humalog Mix\textregistered 25, Novo Mix\textregistered 30):\\} Leveres i ulike blandingsforhold. 10-20 minutter før effekt. Lavest blodsukker ved 1-4 timer etter injeksjon. Ute av kroppen etter 20–24 timer.

				\subsubsection{Tablettbehandling:} 
					Brukes hvis kostbehandling og økt fysisk aktivitet ikke gir tilstrekkelig blodsukkerkontroll.

					\paragraph{Ved overvekt (kroppsmasseindeks(BMI) mer enn 27):\\}Metformin (Glucophage®, Metformin®).

					\paragraph{Ingen overvekt (kroppsmasseindeks(BMI) mindre enn 27):\\}Sulfonylurea(Glibencalmid(Glibenclamid Ratiopharm \textregistered, Glimepirid (Amaryl\textregistered))

					\paragraph{Ved utilfredsstillende effekt av de to nevnt over:\\}Kombinasjonsbehandling, tillegg av sulfonylurea, DPP4 hemmere, GLP-1 analoger, glitazoner og/eller akarbose.

					\paragraph{Hvis man ikke kommer til mål med tabletter:\\} Middels langsomtvirkende insulin i 1–2 daglige doser. Hos overvektige pasienter beholdes metformin og suppleres med middels langsomtvirkende insulin.

					\paragraph{Obs:\\}Noen av de blodsukkersenkende midlene gir fare for hypoglykemi som følge av overdosering, interaksjon (sulfapreparater, antiflogistika, betablokkere, angiotensinkonverterende enzymhemmere, alkohol), dårlig ernæringstilstand, nedsatt leverfunksjon og nyreinsuffisiens. Dette gjelder spesielt midler som stimulerer insulinproduksjonen: Sulfonylurea (Glibenclamid\textregistered, Glimepirid\textregistered, Mindiab\textregistered og Amaryl\textregistered) og glinider (Starlix\textregistered og NovoNorm\textregistered).
		\section{Pasienteksempler}
			Disse pasienteksemplene er ikke tenkt å være typiske men kanskje litt utfordrende. De er anonymiserte pasienter jeg har møtt i legevaktsarbeidet. 
			\subsection{Pasient 11}
				\paragraph{En kvinne i nød...\\}
					Mann, 91 år med lettgradig kognitiv svikt. Har hjemmetjenster gr 3 daglig, mat og medisiner, hjemmehjelp omlag hver 3. uke. Multidose og insulnpenn ggr 1 daglig grunnet diabets type II. Bruker også, Renitec, Selozok, Simvastatin, Albyl-E, Zopiclone, Vesicare, Digoxin og Paracetamol. Han er vanligvis stabil i blodsukker omkring 11 når hjemmetjenesten kommer om kvelden for å sette kveldsdosen. I kveld viser blodsukkeret 4,7 enda han har spist normalt.

					\begin{itemize}
						\item Hvordan tenker du i denne situasjonen og hvordan følges han opp?\\
					\end{itemize}
			\subsection{Pasient 12}
				\paragraph{Et søtt problem...\\}
					En kvinne, 86 år gammel med langtkommen nyresvikt, sterkt nedsatt syn og redusert evne til å stelle seg selv har harr diabetes type II i over 25 år. Hun er plaget med kroniske smerter, mye angst og et blodsukker som vanskelig å regulere. Hun bruker omlag 25 forskjellige medisiner for nyresvikt og hjertesvikt samt diabetes i tillegg til insulin. Hun har pårørende som hjelper i stell men må likevel ha hjemmetjenester 4 ganger om dagen og tilsyn om natten.

				\begin{itemize}
					\item Hva betyr det når blodsukker klokken 22 var 6,3 og klokken 03 er 14,2? Det ble gitt 15 IE lantus kl 22:15.\\ %lantus virker ikke ennå
					\item Hennes HbA\textsubscript{1C} viser en verdi på 10,2 mmol/l. Hva betyr dette for hennes gjennomsnittsblodsukker?
					%blodsukkeret er i gjennomsnitt 33% høyere
				\end{itemize}


