
\documentclass[12pt]{memoir} % use larger type; default would be 10pt


%inkluderte pakker
\usepackage[utf8]{inputenc}
\usepackage{hyperref}
\usepackage{paralist}
\usepackage{graphicx}
\usepackage[margin=8pt,labelfont=bf]{caption}



%slutt på pakkene


\begin{document}
%%Her begynner oversettelsen til norsk
			\renewcommand{\chaptername}{Del}
            \renewcommand{\contentsname}{Innhold}
            \renewcommand\listfigurename{Illustrasjoner}
            \renewcommand\tablename{Tabell}
			\renewcommand\listtablename{Tabeller}
            \renewcommand{\figurename}{Illustrasjon}

%%Her slutter den

\frontmatter
%Grafisk Forside

	\newcommand\nbvspace[1][3]{\vspace*{\stretch{#1}}}
	% allow some slack to avoid under/overfull boxes
	\newcommand\nbstretchyspace{\spaceskip0.5em plus 0.25em minus 0.25em}
	% To improve spacing on titlepages
	\newcommand{\nbtitlestretch}{\spaceskip0.6em}
	\pagestyle{empty}
	\begin{center}
	\bfseries
	\nbvspace[1]
	\Huge
	{\nbtitlestretch\huge
	Forelesningsmateriale\\ Fagakademiet}

	\nbvspace[1]
	\normalsize

	Kliniske problemstillinger i hjemmetjenesten\\
	med pasienteksempler\\
	\nbvspace[1]
	\small av\\
	\Large Pål Ager- Wick\\[0.5em]
	\footnotesize Kommuneoverlege\\

	\nbvspace[2]

	%\includegraphics[width=6in]{./anitascotland.jpg} %placeholder for ete bilde på forsiden
	\nbvspace[3]
	\normalsize

	Tønsberg\\
	\large
	01.12.2013\\
	\nbvspace[1]
	\end{center}

%Slutt grafisk forside

		\chapter{Forord}%!!!
			\paragraph{}
				Dette er forordet
				\\[0.7in]



				DRAMMEN 26.10.2013\\[0.4in]

				Pål CJ Ager-Wick

	\newpage
	\tableofcontents

\mainmatter
	\chapter{Hvordan jobbe systematisk}
		\section{Å lage verktøy for å ikke være redd}
		\section{Lær av feil}
		\section{Hvordan finne god informasjon}
		\section{Sjekklister fra bakken og opp?}
		\section{Systemer er kjedelig}
		\section{Noen praktiske prinsipper}
	\chapter{Hjertesykdommer}
		Her er noen få av alle hjertesykdommene forklart. Dette er ment å være et tillegg til forelesningen slik at man ikke må notere så mye. Dette er ikke en fullstendig oversikt over hjertesykdommene. Det er også forsøkt å forklare på et så enkelt som mulig nivå slik at hjertespesialister eller andre vil føle at det er litt enkelt. Dette maerialet er ikke laget for dem.\\
		\section{Anatomi}
			\paragraph{Slik ser hjerte ut\\}
				%!!! Her må et bilde inn
			\paragraph{Effektiv jobbing\\}
				Når hjertet pumper med vanlig frekvens er det veldig effektivt. Da sørger det for jevn transport av blodet på en mest mulig energieffektiv måte. Når hjerteslagene blir veldig raske blir hjertet mindre effektivt\cite{!!!}. Man kan kalle det en funksjonell hjertesvikt. Det betyr at hjertet er friskt, men jobber på- eller over grensen for at blodet skal strømme fritt.
				%legg inn video an ultralyd ecco cor!!!
		\section{Fysiologi}	
			\paragraph{Hva menes med hjerte- og karsykdommer?\\}
				Som alle organer i kroppen har hjertet sine egne blodårer. De er ekstra utsatt for åreforkalking, eller atherosklerose som det heter på latin. Atherosklerose er kalkinnlagring i blodåreveggen som gjør den stiv og samtidig klumpete på innsiden. Dette hindrer blodgjennomstrømningen. Mengden med kalk i blodårene er varierende gjennom livet, men fet mat, høyt blodtrykk og sigaretter gjør at mer kalk lagres. 
		\section{Patologi}	
			\paragraph{Skader oppstår\\}
				Noen steder blir blodpassasjen dårlig og det kan dannes skader fordi vevet ikke får nok oksygen. Hjertet er en muskel med innebygget nervesystem og noen ganger blir det små skader som gror til arr i forkammeret. Dette kan skape atrieflimmer\cite{!!!}. Hvis blodårene tetter seg rundt hjertekammeret får man ofte anginasmerter, og dersom blodåren blir helt tett er det et infarkt.
		\section{Klinikk}
			\subsection{Hjerteinfarkt}	
				\paragraph{Symptomer\\}
					Trykkende smerte i brystet, utstråling til venstre arm eller underkjeve. Blek og kaldsvett, klam og tungpusten. Dette er noen klassiske symptomer ved hjerteinfarkt. Vi må passe oss fordi, eldre, pasienter med diabetes eller kvinner har ofte helt andre symptomer. 
				\paragraph{Førstehjelp\\}
					Ring ambulansen, vær hos pasienten. Gi oksygen og Dispiril hvis dere har. 
				\paragraph{Farlige momenter\\}
					De som dør av hjerteinfarkt får ofte akutt hjertflimmer. Dette er ikke atriflimmer, men kammerflimmer og er helt forskjellig. Hjertet slår med 300 slag i minuttet. Pasienten er bevisstløs og den eneste redningen er å bruke hjertestarter og å gjøre hjerte- lungeredning. 
				\paragraph{Hva skjer etterpå?\\}
					Alle pasienter som har hatt hjerteinfarkt får nesten samme type medisiner:
					\begin{itemize}
						\item Metoprolol(SelZok(R)), gjør at hjertet for "hvile". Forebygger nye infarkt og hjerterytmeforstyrrelser. Senker blodtrykket, og gjør at makspulsen blir lavere ved fysiske anstrengelser. Noen menn blir impotente. Kalles også "Betablokker"\\
						\item Acetylsalisylsyre\\%!!!
					\end{itemize}
				\paragraph{Hva må hjemmetjenesten være oppmerksomme på?\\}

			\subsection{Hjertesvikt}
			\subsection{Atrieflimmer}
			\subsection{Digitalis}
		\section{Pasienteksempler}
			\subsection{Pasient 1}
			\subsection{Pasient 2}
	\chapter{Nevrologiske sykdommer}
		\section{Anatomi}
		\section{Fysiologi}	
		\section{Patologi}
		\section{Klinikk}
		\section{Pasienteksempler}
			\subsection{Pasient 3}
			\subsection{Pasient 4}
	\chapter{Lungesykdommer}
		\section{Anatomi}
		\section{Fysiologi}
		\section{Patologi}
		\section{Klinikk}
		\section{Pasienteksempler}
			\subsection{Pasient 5}
			\subsection{Pasient 6}
	\chapter{Urinveier}
		\section{Anatomi}
		\section{Fysiologi}
		\section{Patologi}
		\section{Klinikk}
		\section{Pasienteksempler}
			\subsection{Pasient 7}
			\subsection{Pasient 8}
	\chapter{Delir}
		\section{Anatomi}
		\section{Fysiologi}
		\section{Patologi}
		\section{Klinikk}
		\section{Pasienteksempler}
			\subsection{Pasient 9}
			\subsection{Pasient 10}
	\chapter{Diabetes i alderdommen}
		\section{Anatomi}
		\section{Fysiologi}
		\section{Patologi}
		\section{Klinikk}
		\section{Pasienteksempler}
			\subsection{Pasient 11}
			\subsection{Pasient 12}

 
%Ordforklaringer?
%Stikkordsliste

 \renewcommand{\bibname}{Kilder:}
              \begin{thebibliography}{99}

              \end{thebibliography}


\end{document}